\documentclass[12pt,spanish,twoside]{thesis}

\usepackage[T1]{fontenc}
\usepackage[utf8]{inputenc}
\usepackage[spanish, es-tabla]{babel}
\usepackage{amsmath}

%Interlineado
\usepackage{setspace}
\setstretch{1.5}

\usepackage{times}
\usepackage{amssymb}
\usepackage{float}
\usepackage{color}
\usepackage{graphicx}
\usepackage{eso-pic}
\usepackage{multicol}
\usepackage{enumerate}
\usepackage{url}
\usepackage{soul}
\usepackage{fancyhdr}
\usepackage[raggedright]{titlesec}

% profundidad de tabla de contenidos
\setcounter{secnumdepth}{4}
\setcounter{tocdepth}{4}

%Margenes
\usepackage[top=3cm,bottom=3cm,inner=4.2cm,outer=3cm]{geometry}
\pagestyle{empty} 
\frenchspacing

\fancyhead[L]{}
\fancyhead[C]{}
\fancyhead[R]{}
\fancyfoot[RO]{\thepage}
\fancyfoot[LE]{\thepage}
\fancyfoot[C]{}

\newcommand{\mychapter}[2]{
    \setcounter{chapter}{#1}
    \setcounter{section}{0}
    \chapter*{#2}
    \addcontentsline{toc}{chapter}{#2}
	}

% Modificar número de capítulo en subsección
\renewcommand{\thesection}{}
\renewcommand{\thesubsection}{\arabic{chapter}.\arabic{subsection}}

% cambiar cuadro por tabla, y fig por figura.
\renewcommand{\tableshortname}{Tabla}
\renewcommand{\figureshortname}{Figura}

\makeatletter
\def\@seccntformat#1{\csname #1ignore\expandafter\endcsname\csname the#1\endcsname\quad}
\let\sectionignore\@gobbletwo
\let\latex@numberline\numberline
\def\numberline#1{\if\relax#1\relax\else\latex@numberline{#1}\fi}
\makeatother

%Fin Preambulo
%%%%%%%%%%%%%%%%%%%%%%%%%%%%%%%%%%%%%%%%%%%%%%%%%%%%%

\begin{document}
\thispagestyle{empty}

\begin{center}
\linespread{1.15}
\textbf{\large{UNIVERSIDAD TÉCNICA FEDERICO SANTA MARÍA\\}
\normalsize{DEPARTAMENTO DE ELECTRÓNICA\\VALPARAÍSO - CHILE\\}}

\vspace{0.5cm}
\begin{figure}[H]
\centering
  \includegraphics[width=5.85cm]{fig/usmLogo}
\end{figure}
\vspace{0.5cm}

\linespread{1}\hangindent=0cm
\textbf{\Large ``TITULO DE LA MEMORIA''\\}
\vspace{3cm}

\hangindent=0cm\large \textbf{NOMBRE DEL ESCLAVO}\\
\vspace{0.5cm}
\hangindent=0cm\normalsize \textbf{MEMORIA DE TITULACIÓN PARA OPTAR AL TÍTULO DE INGENIERO CIVIL TELEMÁTICO}\\
\vspace{1cm}
\hangindent=0cm\normalsize \textbf{PROFESOR GUIA: \hspace{2cm} PROFE 1.}\\
\vspace{0.5cm}
\hangindent=0cm\normalsize \textbf{PROFESOR CORREFERENTE: \hspace{2cm} PROFE 2.}\\
\vspace{2cm}
\hangindent=0cm\normalsize \textbf{MES - A\~NO}\\

\end{center}
\thispagestyle{empty}

%% pagina post portada
\newpage 
\thispagestyle{empty}

\renewcommand\headrulewidth{0pt}

%% cap0, agradecimientos, resumen ...
\frontmatter
\pagestyle{fancy}
\addcontentsline{toc}{chapter}{Agradecimientos}
\section*{\qquad \qquad \qquad \qquad Agradecimientos}

Agradecimientos ``por las risas y el vacilon''.

\cleardoublepage
\thispagestyle{empty}
\vspace*{\stretch{12}}
\begin{flushright}
\itshape
la dedicatoria
\end{flushright}
\vspace{\stretch{3}}


\cleardoublepage \newpage
\addcontentsline{toc}{chapter}{Resumen}
\section*{\qquad \qquad \qquad \qquad \qquad Resumen}


\cleardoublepage \newpage
\addcontentsline{toc}{chapter}{Abstract}
\section*{\qquad \qquad \qquad \qquad \qquad Abstract}


\cleardoublepage \newpage
\addcontentsline{toc}{chapter}{Glosario}
\section*{\qquad \qquad \qquad \qquad Glosario}


% Para modificar, por ejemplo, que en el índice el nombre del capítulo aparezca contínuo "Capitulo X:",
%	se debe editar directamente el archivo main.toc
\tableofcontents
\listoffigures
\listoftables

%% cap1 ... -> contenido principal
\mainmatter
\pagestyle{fancy}

\newpage

\mychapter{1}{Capítulo 1}
\section{1. Introducción}
\subsection{subsección}

Como se menciona en \cite{method}\cite{modnegweb}\cite{modnegweb2}

Como se aprecia en la figura \ref{usm}

\begin{figure}[H]
\centering
\includegraphics[scale=1]{fig/usmLogo.png}
\caption{Logo USM}
\label{usm}
\end{figure}

\subsection{subsección}

\subsubsection{subsubsección}

\mychapter{2}{Capítulo 2}
\section{2. nombre capítulo 2}
Bla bla ...

\subsection{subsección 1 capítulo 2}

Bla bla ... en la tabla \ref{tab:xx}

\begin{table}[H]
\begin{center}
\begin{tabular}{c c} \hline
nombre & valor \\ \hline \hline
nombre1 & valor1 \\
nombre2 & valor2 \\
nombre3 & valor3 \\ 
nombre4 & valor4 \\
nombre5 & valor5 \\
.. & .. \\ \hline
\end{tabular}
	\caption{Nota al pie de tabla.}
	\label{tab:xx}
\end{center}
\end{table}

\subsection{subsección 2 capítulo 2}
Bla ... Blah\footnote[1]{Sitio web de google, [consulta: XX Mes A\~no] \\
 \url{http://www.google.com}}... Bla ... 

\include{referencias}

		
\end{document}
